%!TEX root = main-doc.tex
%
% File: validationofresults.tex
%
% Date: ? 
%
% Description:
%   The background is given to ...
%
%
\chapter{Validation of Results} \label{chap:validationofresults}
\vspace{-1cm}
%\summary{Give chapter summary}

%%%%%%%%%%%%%%%%%%%%%%%%%%%%%%%%%%%%%%%%%%%%%%%%%%%%%%%%%%%%%%%%%%%%%%%%%%%%%%%%
%\section{Data Warehousing}
In order to test the performance and efficiency of the proposed solution, several tests will be run. Experiments similar to the ones done in \cite{otoo:2013:ced}, \cite{pedereira:2015:cas}, \cite{otoo:2016:msa} and \cite{goil:bess} and will be done where three datasets of different sizes, three datasets of different dimensions (2D,3D and 4D) and three datasets of different sparsity levels (i.e. 80\%, 90\% and 95\%) will be stored in the XSAS representation (i.e. six different datasets). Three different queries will be run and timed on the datasets, these queries will be run three times each in order to ensure that an accurate average measurement has been taken. An assessment of the storage utilization (in percentage) will be done on the arrays. A further investigation will be conducted on 2D arrays comparing the XSAS representation with BESS and the XSAS representation with CRS.