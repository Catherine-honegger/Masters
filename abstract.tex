%!TEX root = main-doc.tex

\begin{abstract}

Data warehousing is a method used to combine multiple varied datasets into one complete and easily manipulated database as a decision support system. Research over several years has concluded that multidimensional representation of data in data warehousing not only gives a good visual perspective of the data to the user, but also provides a storage scheme for efficient processing. Since data in a data warehouse grows dynamically the multidimensional representation should also be expanded dynamically. Efficient dynamic storage schemes for storing dense, extendible, multidimensional arrays by chunks have been developed \cite{nimako:2012:ced,pedereira:2015:cas}. However in data warehousing the corresponding multidimensional arrays are predominantly sparse arrays. Currently no storage or mapping techniques allow for the extendibility of multidimensional sparse arrays \cite{nimako:2016:cea}. This research proposes to improve the modelling and representation of extendible multidimensional sparse arrays so that useful compression and storage efficiency can be determined. Our work addresses extendibility of the sparse array only, and does not concern the shrinking of the sparse array.
%150 words - purpose, research methods, procedures employed, major results, conclusions, reccomendations - Start with a statement of the major theme of the dissertation
%1 sentence motivation
%1 sentence done before (related work) relevance, limitation of related works
%Main objective(Main results expected)
%limitation of your work

\end{abstract}
