%!TEX root = main-doc.tex
%
% File: preliminaryresults.tex
%
% Date: ? 
%
% Description:
%   The background is given to ...
%
%
\chapter{Preliminary Results} \label{chap:preliminaryresults}
\vspace{-1cm}
%\summary{This chapter provides an outline of any preliminary experiments that have been conducted on the extendibility of multidimensional sparse arrays.}

%%%%%%%%%%%%%%%%%%%%%%%%%%%%%%%%%%%%%%%%%%%%%%%%%%%%%%%%%%%%%%%%%%%%%%%%%%%%%%%%
\section{Investigations on Pre-existing Technologies}
Some preliminary investigations have been conducted on pre-existing Big Data formats and pre-existing data analysis platforms. A quick overview of the TileDB storage manager shows that the storage manager only caters for 2D sparse arrays and does not cater for higher dimensions. TileDB inserts new data into sparse arrays by storing the new data in previously empty cells \cite{tiledb:tm101}. A look at two data analysis platforms, Weka and RapidMiner, show how pre-existing technologies cater for MOLAP operations, however the platforms do not allow for any extendibility of the actual datasets they are operating on.

%%%%%%%%%%%%%%%%%%%%%%%%%%%%%%%%%%%%%%%%%%%%%%%%%%%%%%%%%%%%%%%%%%%%%%%%%%%%%%%%
\section{XSAS Implementation in C}
Some coding has been conducted in C making use of the GNU Scientific Library. The example matrix given in Figure \ref{fig:exampleMatrix} was stored using CRS. The program is currently being extended to store the matrix using BESS. Once the BESS algorithm has been implemented the code will be extended to incorporate PATRICIA Trie indexing. Upon completion of the PATRICIA Trie indexing the density extendibility and the bounds extendibility algorithms will be implemented and their performances analysed.