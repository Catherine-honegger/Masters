%!TEX root = main-doc.tex
%
% File: schedule.tex
%
% Date: ? 
%
% Description:
%   The background is given to ...
%
%
\chapter{Schedule and Time-Line for Completion} \label{chap:schedule}
\vspace{-1cm}
%\summary{This chapter details the research tasks and their proposed date of completion.}

%%%%%%%%%%%%%%%%%%%%%%%%%%%%%%%%%%%%%%%%%%%%%%%%%%%%%%%%%%%%%%%%%%%%%%%%%%%%%%%%
\section{Proposed Task Completion Time-Line}
The research is expected to be completed by April 2018. Regular meetings are held with the supervisor once a week. There is an open door policy enabling casual meetings to be arranged  ensuring that there is constant feedback throughout the lifespan of the proposed research. 

Table \ref{tab:timeline} presents a set of major tasks along with their preliminary completion dates for the proposed work.

\begin{table}[H]
	\caption{Proposed Task Completion Times.\label{tab:timeline}}
	\begin{center}
		\begin{tabular}{|l|l|}
		\hline 
		\textbf{Task Description} & \textbf{Proposed Completion Date} \\ 
		\hline 
	 	Begin Research Project & January 2017\\
	 	Literature Review of Data Warehousing & March 2017\\
	 	Literature Review of Sparse Array Representation & May 2017\\
	 	Literature Review of Dynamic Dense Array Representation & July 2017\\
	 	Documentation of the Methodology & September 2017\\
	 	Submit Research Proposal for Approval & October 2017\\
	 	Visual Presentation of the Research Proposal & October 2017 \\
	 	Develop and Implement Storage and Chunking Techniques & December 2017\\
	 	Develop Dynamic Expansion Algorithms & February 2018\\
		Evaluate Performance and Analyse Results & April 2018\\
		Submit Research Dissertation for Approval & April 2018\\
		\hline 
		\end{tabular} 
	\end{center}
\end{table}