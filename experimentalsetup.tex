%!TEX root = main-doc.tex
%
% File: experimentalsetup.tex
%
% Date: ? 
%
% Description:
%   
%
%
\chapter{Experimental Setup} \label{chap:experimentalsetup}
\vspace{-1cm}
%\summary{This chapter details the computational environment and the experimental data in order to allow for independent reproducibility and confirmation of the conducted experiments.}

%%%%%%%%%%%%%%%%%%%%%%%%%%%%%%%%%%%%%%%%%%%%%%%%%%%%%%%%%%%%%%%%%%%%%%%%%%%%%%%%
%\vspace{-1cm}
\section{Computational Environment}

%%%%%%%%%%%%%%%%%%%%%%%%%%%%%%%%%%%%%%%%%%%%%%%%%%%%%%%%%%%%%%%%%%%%%%%%%%%%%%%%
%\subsection{Experimental Computational Model and Software Environment}
The experiments will be run on a laptop with an Intel(R) Core(TM) i5-2450 multi-core processor at 2.50~GHz and 4~GB of memory running Ubuntu 64-bit Linux 16.04.3 LTS. We have the capability of using a computer with 24~GB of memory for medium sized data and machines with up to 128~GB for extremely large datasets.

The experiments will be implemented in C using GNU Compiler Collection (GCC) 7.2. The code development process will make use of software engineering techniques and best practice with regards to ensuring code is efficient and hardware resources are utilized correctly. This involves making use of tests, ensuing the code is non-repetitive, using standard libraries where possible, and good commenting and naming conventions. The source code will be composed in a revision control environment using Git.

%%%%%%%%%%%%%%%%%%%%%%%%%%%%%%%%%%%%%%%%%%%%%%%%%%%%%%%%%%%%%%%%%%%%%%%%%%%%%%%%
%\section{Experimental Data}

%\textbf{NB - Still working on this section}

%%%%%%%%%%%%%%%%%%%%%%%%%%%%%%%%%%%%%%%%%%%%%%%%%%%%%%%%%%%%%%%%%%%%%%%%%%%%%%%%
%\section{Software Engineering Practice}

%\textbf{NB - Still working on this section}